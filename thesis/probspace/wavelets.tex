\section{Wavelet}

Wavelet compression uses sets of complementary wavelets transforms to decompose
a signal, representing it by the coefficients of mathematical functions.
Typically, the wavelet transform approximates the signal meaning that in order
to achieve lossless compression the residuals between the approximation and the
actual signal must be encoded. This encoding is more compressible if the
approximation is accurate. Furthermore, the coefficients must also be recorded
in order to reconstruct the wavelet transform during decompression.

Discrete wavelet transforms (DWT) is the most commonly used wavelet transform in
signal compression because it can be implemented naturally using a computer
since it discretely samples each wavelet. There are several DWT forms namely
Haar, Daubechies and undecimated.
