In this chapter we will design several novel lossless compression methods for
nanopore signal data.
Firstly, we begin by presenting a more appropriate container for downstream
compression known as vbbe21 -- with the intention to replace the role of svb.
Then, we estimate the effect of applying Huffman coding and range coding to the
one byte data of vbbe21.

Attempts to exploit the characteristics of the signal data follow. Optimal and
approximating subsequence searching algorithms are presented and analysed.
Encoding the stall separately is then explored, followed by a partitioning of the
data into three types of sequences: jumps, falls and flats.

All the methods presented are implemented in the GitHub repository available
here: \url{https://github.com/sashajenner/honours}. The benchmarking program is
also provided which can be used on any nanopore signal data set as long as it's
in the SLOW5 file format.
