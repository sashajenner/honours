The primary research problem is to develop an improved lossless encoding strategy for nanopore signal data. This requires some unpacking and can be sub-divided into the following smaller research problems.

\subsubsection{Determine the salient features of nanopore signal data}
The first problem is to determine the features and repeated patterns of nanopore signal data. Understanding the data is the most crucial step in developing a data-specific lossless encoder. Once the features have been recognised, they can be directly exploited by an encoding strategy in order to reduce the data's redundancy and bit rate. The intention is to encode each nanopore \textit{read} separately in order to maintain random parallel access. With this in mind, there are potentially patterns between independent reads which should first be determined.

\subsubsection{Apply and evaluate appropriate existing encoding strategies}
The next problem involves experimenting with and evaluating existing lossless encoding strategies which are suitable for nanopore signal data. Most likely, there already exists a strategy in the literature which when applied to nanopore data would prove better than the state-of-the-art. This problem involves further investigation into the literature of integer and signal data compression. It also requires developing a iterative evaluation framework which can quickly determine whether an existing strategy implementation is worth further investigation. Furthermore, it may prove that an existing strategy forms the basis for an improved encoder when modified specifically with the salient features of nanopore signal data in mind.

\subsubsection{Develop a new lossless algorithm (or modify an existing one) which is better than VBZ}
This is the final problem and main contribution I intend to make. It is clear from the literature review that few lossless encoders specific to nanopore signal data have been considered. A new lossless algorithm which is better than the state-of-the-art known as VBZ (\url{https://github.com/nanoporetech/vbz_compression/}) would significantly alleviate the storage issues and/or long analysis times faced by the nanopore sequencing community.
