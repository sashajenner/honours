\subsubsection{Signal compression}

A \textit{wavelet} is a brief oscillation which starts and ends with an
amplitude of zero. Sets of complementary wavelets are desired for wavelet-based
compression since they can reversibly decompose a signal. Wavelet transforms are
more useful than Fourier transforms for approximating signal data with sharp
peaks and non-periodic behaviour. Wavelet compression is most applicable to
image data but has been successfully applied to audio and video data as well as
electrocardiograph (ECG) signals \cite{ecg}. It is used in the JPEG 2000 standard.

Wavelet compression decomposes a signal into the coefficients of mathematical
functions.
Typically, the wavelet transform approximates the signal meaning that in order
to achieve lossless compression the residuals between the approximation and the
actual signal must be encoded. This encoding is more compressible if the
approximation is accurate. Furthermore, the coefficients must also be recorded
in order to reconstruct the wavelet transform during decompression.

Discrete wavelet transforms (DWT) is the most commonly used wavelet transform in
signal compression because it can be implemented naturally using a computer
since it discretely samples each wavelet. There are several DWT forms namely
Haar, Daubechies and undecimated.
