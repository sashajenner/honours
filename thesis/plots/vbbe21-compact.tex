\begin{figure}
\centering\begin{tikzpicture}[node distance=0cm,start chain=1 going right] \footnotesize
  \tikzstyle{mytape}=[draw,minimum height=1.7cm]
	\node(A1)  [on chain=1,mytape,fill=blue!20] {$\underset{\text{of exceptions}}{\underbrace{\overbracket{b_e}^{\text{4 bits}}}_{\text{bits for number}}}$};
	\node(A2)  [on chain=1,mytape,fill=blue!20] {$\underset{\text{exceptions}}{\underbrace{\overbracket{\text{ }e\text{ }}^{b_e\text{ bits}}}_{\text{number of}}}$};
	\node(B1)  [on chain=1,mytape,fill=yellow!20] {$\underset{\text{position}}{\underbrace{\overbracket{b_p}^{5\text{ bits}}}_{\text{bits per exception}}}$};
	\node(B2)  [on chain=1,mytape,fill=yellow!20] {$\underbrace{\overbracket{p1,\delta(p_1,p_2,\dots,p_e)-1}^{b_pe\text{ bits}}}_{\text{exception positions}}$};
	\node(C1)  [on chain=1,mytape,fill=yellow!35] {$\underset{\text{byte exception}}{\underbrace{\overbracket{b_x}^{4\text{ bits}}}_{\text{bits per two}}}$};
	\node(C2)  [on chain=1,mytape,fill=yellow!35] {$\underbrace{\overbracket{(x_{p_1},x_{p_2},\dots,x_{p_e})-256}^{b_xe\text{ bits}}}_{\text{two byte exceptions}}$};
	\node [on chain=1,mytape,fill=gray!35] {$\underbrace{\overbracket{\text{ }\dots \text{ }}^{<1\text{ byte}}}_{\text{padding}}$};
	\node(D)  [on chain=1,mytape,fill=green!35] {$\underbrace{\overbracket{x_{q_1},x_{q_2},\dots,x_{q_{n-e}}}^{n-e\text{ bytes}}}_{\text{one byte data}}$};
\end{tikzpicture}
	\caption{\label{fig:vbbe21-compact}The compact vbbe21 takes $n$ unsigned
	16-bit integers and finds those which cannot fit into one byte. These
	are encoded by bit packing the number of exceptions, the deltas of the
	exceptions' positions and the two byte exceptions subtracted by 256.
	Less than one byte is used for padding to align the bit packed data to
	the next byte boundary. Then, the one byte data is recorded as in
	vbe21.}
\end{figure}
