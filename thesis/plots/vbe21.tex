\begin{figure}
\centering\begin{tikzpicture}[node distance=0cm,start chain=1 going right] \footnotesize
  \tikzstyle{mytape}=[draw,minimum height=1.5cm]
	\node(A1)  [on chain=1,mytape,fill=blue!20] {$\underset{\text{exceptions}}{\underbrace{\overbracket{\text{ }e\text{ }}^{\text{2 bytes}}}_{\text{number of}}}$};
	\node(A2)  [on chain=1,mytape,fill=yellow!20] {$\underbrace{\overbracket{p_1,p_2,\dots,p_e}^{4e\text{ bytes}}}_{\text{exception positions}}$};
	\node(A3)  [on chain=1,mytape,fill=yellow!35] {$\underbrace{\overbracket{x_{p_1},x_{p_2},\dots,x_{p_e}}^{2e\text{ bytes}}}_{\text{two byte exceptions}}$};
	\node(A4)  [on chain=1,mytape,fill=green!35] {$\underbrace{\overbracket{x_{q_1},x_{q_2},\dots,x_{q_{n-e}}}^{n-e\text{ bytes}}}_{\text{one byte data}}$};
\end{tikzpicture}
	\caption[The vbe21 encoding.]{\label{fig:vbe21} The vbe21 encoding takes two byte integers
	$x_1,x_2,\dots,x_n$ and encodes those which cannot fit into one byte as
	\textit{exceptions} at the beginning of the stream. There are $e$
	exceptions which are recorded by their original positions
	$p_1,p_2,\dots,p_e$ and values $x_{p_1},x_{p_2},\dots,x_{p_e}$.
	Following this is the regular one byte data where $q_i$ is the original
	position of the $i$-th one byte data point. This is beneficial when
	there are few exceptions in the data.}
\end{figure}
